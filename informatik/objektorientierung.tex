\section{Objektorientierung}
\subsection{Klassen}

Die Objektorientierte Programmierung mit Klassen hat zum Ziel,
reale Zusammenhänge zu modellieren und Dinge zu strukturen und
zu organisieren. Dazu gibt es Klassen mit Attributen und Methoden
mit denen sich Objekte (Instanzen der Klasse) erzeugen lassen.
Eine Klasse ist ein Schemata zur Erzeugung eines Objekts.

\pgfsetlayers{package0, main}

\vspace*{0.5cm}
\begin{figure}[H]
\begin{center}
\hspace*{-2cm}
\begin{minipage}{.7\textwidth}
    \begin{lstlisting}[language=python]
    class Car:
        def __init__(self, name: str, speed: float):
            self.name = name      # Attribut
            self.speed = speed
            self.__km_driven = 0
            
        def getDriven() -> float: # Methode
            return self.__km_driven
            
        def drive(hours: float):
            self.__km_driven += self.speed * hours
    \end{lstlisting}
    \caption{Klasse in Python}
\end{minipage}
\begin{minipage}{.3\textwidth}
    \begin{tikzpicture}
        \umlclass{Auto}{ 
        + name : Zeichenkette \\ 
        + speed : Dezimalzahl \\ 
        - km\_driven : Dezimalzahl
        }{ 
            + getDriven() : Dezimalzahl \\
            + drive(hours: Dezimalzahl) :
            }
    \end{tikzpicture}
    \caption{UML Klassendiagramm}
\end{minipage}
\end{center}
\end{figure}

\subsection{Vererbung}
\subsection{Klassendiagramme}