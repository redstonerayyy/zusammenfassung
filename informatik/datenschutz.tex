\section{Datenschutz}

Durch die Digitalisierung werden immer mehr Daten gesammelt und verarbeitet.
Damit dies rechtmäßig und Gesetzeskonform geschieht und die Rechte der Bürger
gewahrt werden, gibt es Regeln, Gesetze und Grundideen des Datenschutzes.
In der EU gibt es deshalb die Datenschutzgrundverordnung.

\subsection{Grundbegriffe des Datenschutzes}

\subsubsection{Personenbezogene Daten}

Personenbezogene Daten sind alle Daten, die einer Person werden können,
z. B. durch Name, Kennnummern, Standortdaten oder einer Online-Kennung.
Daten der physischen, physiologischen, genetischen, psychischen, wirtschaftlichen,
kulturellen oder sozialen Identität einer Person werden dann zu personenbezogenen Daten.

\subsubsection{Informationelle Selbstbestimmung}

Jeder hat das Recht darüber zu bestimmen, welche persönlichen Daten von anderen Akteuren
gespeichert und verarbeitet werden dürfen. Dabei kann eine Einwilligung jederzeit
zurückgenommen werden. Dieses Recht wird dabei nur durch andere Gesetze eingeschränkt.

\subsubsection{Prinzipien des Datenschutzes}

Das Prinzip der Datensparsamkeit sagt aus, dass nur notwendige Daten erhoben werden dürfen.

Das Prinzip der Zweckbindung sagt aus, dass Daten nicht für andere Zwecke als vorher
vereinbart genutzt werden dürfen.

Das Prinzip der Transparenz sagt aus, dass Betroffenen Auskunft gegeben werden muss,
welche Daten verarbeitet werden.

Das Prinzip der Datenrichtigkeit sagt aus, dass Sorge getragen werden muss,
dass alle personenbezogenen Daten jederzeit korrekt sind.

Das Prinzip der Datenlöschung sagt aus, dass personenbezogene Daten nur solange
wie nötig gespeichert werden dürfen und danach gelöscht oder anonymisiert/pseudonymisiert
werden müssen.

Das Prinzip der Datenvertraulichkeit sagt aus, dass personenbezogene Daten
vor unrechtmäßiger Verarbeitung geschützt werden und vor unerlaubtem
Zugriff geschützt sein müssen.
