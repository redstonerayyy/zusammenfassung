\section{Kryptologie}
\subsection{Verschlüsselung im Alltag}

Verschlüsselung ist ein Grundpfeiler der modernen Kommunikation und absolut nötig,
damit kein Unbekannter und Unbefugter private Nachrichten oder Passwörter mitlesen
kann. Ohne starke Verschlüsselungsverfahren, wäre die weitergabe sensibler Daten über
das Internet nicht möglich.

\subsection{Historische Verschlüsselungsverfahren}

\subsubsection{Caesar Chiffre}

Schon die alten Römer und viele nach ihnen erkannten, dass es durchaus von Vorteil
ist, wenn die Feinde nichts über den geheimen Schlachtplan oder die nächste Intrige
erfahren können.
Das sogenannte Caesar-Chiffre ist die einfachste Form eines symetrischen,
monoalphatischen Verschlüsselungsverfahren für Texte.
Man bestimmt eine Verschiebung zwischen 1 und 25 und ersetzt dann
alle Buchstaben des Klartexts, um den Geheimtext zu erhalten.
Kennt man die Verschiebung, so lässt sich die Nachricht auch wieder entschlüsseln.
Dieses Verfahren ist allerdings nicht sicher, weswegen es heutzutage niemand verwendet.
Um einen Buchstaben zu verschlüsseln sucht man diesen im oberen Alphabet und
ersetzt diesen dann durch den Buchstaben darunter aus dem unteren Alphabet.
Beim Entschlüsseln ist dies umgekehrt. Man ersetzt den Buchstaben im Geheimtext
durch den Buchstaben aus dem oberen Alphabet, der über der Position des Buchstabens
im unteren Alphabet ist.

\begin{lstlisting}
A B C D E F G H I J K L M N O P Q R S T U V W X Y Z
Y Z A B C D E F G H I J K L M N O P Q R S T U V W X

Klartext:   INFORMATIK
Geheimtext: GLDMPKYRGI
\end{lstlisting}

\subsection{Symetrische Verschlüsselung}

Symetrische Verschlüsselungsverfahren zeichnen sich dadurch aus, dass zum
Ent- und Verschlüsseln der gleiche Algorithmus und der gleiche Schlüssel verwendet
wird. Beispiele sind die Caesar Chiffre, die Vigenere Chriffe oder das Blockchiffre AES-256.

\subsubsection{Blockchriffen}

\subsection{Asymetrische Verschlüsselung (Public Key Cryptography)}

\subsubsection{Diffie und Hellmann Verfahren}

\subsubsection{RSA und Schlüsselpaare}

\subsection{Public Key Infrastructure (CA-Certificates)}

\subsubsection{Zertifikate}

\subsubsection{Signaturen}
