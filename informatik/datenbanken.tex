\section{Datenbanken}

\subsubsection{Begriffe bei Datenbanken}

<Todo: Primärschlüssel, attribut, fremschlüssel, tabelle, einfüge, änderungs, löschanomalie>


\subsection{Entity-Relationship-Modelle (ER-Modelle)}

Ein Entity-Relationsip Modell soll Entitäten und Beziehungen eines Sachverhalts
darstellen und so z. B. die Erstellung einer Datenbank erleichtern.
Es gibt mehrere Arten zur Notation von Entity Relationship Modellen,
in diesem Fall wird die Chen Notation betrachtet.

\vspace*{1cm}

\adjustbox{scale=0.8}{
\begin{tikzpicture}[auto,node distance=1cm]
\node[entity] (node1) {Student}[grow=up,sibling distance=3.5cm]
    child[text width=2cm, align=center, level distance=2cm] {node[attribute] {Lastname}}
    child[text width=2cm, align=center, level distance=2cm] {node[attribute] {Firstname}}
    child[text width=2cm, align=center, level distance=2cm] {node[attribute] {\underline{ID}}};

\node[relationship] (rel1) [below = of node1] {Takes};
\node[attribute] (node4) [left = of rel1] {Hours};
\path (rel1) edge (node4);

\node[entity] (node2) [below = of rel1]{Class}[grow=down,sibling distance=3.5cm]
    child[text width=2cm, align=center, level distance=2cm] {node[attribute] {\underline{ID}}}
    child[text width=2cm, align=center, level distance=2cm] {node[attribute] {Subject}}
    child[text width=2cm, align=center, level distance=2cm] {node[attribute] {Teacher}};
\path (rel1) edge node {n} (node1) edge node {m}(node2);

\node[relationship] (rel2) [right = of node1] {Goes To};

\node[entity] (node3) [right = of rel2]{School}[grow=right, sibling distance=2.5cm]
    child[text width=2cm, align=center, level distance=2cm] {node[attribute] {\underline{ID}}}
    child[text width=2cm, align=center, level distance=2cm, xshift=2cm] {node[attribute] {Name}}
    child[text width=2cm, align=center, level distance=2cm] {node[attribute] {Type}};

\path (rel2) edge node {m} (node1) edge node {1}(node3);
\end{tikzpicture}
}

\vspace*{0.3cm}

\subsubsection{Erklärung der ER-Notation nach Chen}

\vspace*{0.3cm}

\begin{enumerate}
    \item Ein Rechteck stellt eine Entität dar.
    \item Ein Oval stellt ein Attribut einer Entität dar.
    \item Ein unterstrichenes Attribut ist Teil des Primärschlüssels.
    \item Eine Raute erläutert die Beziehung zweier Entitäten.
    \item Ein Oval an einer Raute kann der Beziehung Attribute zuordnen.
    \item An einer Beziehung zwischen zwei Entitäten sind die Kardinalitäten notiert.
\end{enumerate}

\clearpage

\subsubsection{Kardinalitäten}

Die Kardinälität einer Beziehung beschreibt, wie vielen Entitäten der anderen
Entität zugeordnet werden können. Beispielsweise kann ein Schüler nur auf 1 Schule 
gehen, aber auf eine Schule können viele Schüler gehen. Klassen besucht ein Schüler
hingegen mehrere und in eine Klasse gehen mehrere Schüler. Es gibt folgende Kardinalitäten,
wobei man bei jeder Kardinalität auch angeben kann, ob einer Zuordnung optional ist
oder wie viele Zuordnung es geben muss.

\begin{table}[H]
    \begin{tabular}{|l|l|l|}
    \hline
        Kardinälität & Entität A & Entität B \\ \hline
        Eins zu Eins & 1 & 1 \\ \hline
        Eins zu Vielen & 1 & m \\ \hline
        Viele zu Eins & m & 1 \\ \hline
        Viele zu Viele & m & n \\ \hline
        \end{tabular}
\end{table}

\subsection{Normalformen}

Bei der Erstellung eines Datenbankschemas in einer Relationellen Datenbank
(z. b. aus einem ER-Modell) unterscheidet man, je nach Eigentschaften des
Schematas, in welcher Normalform sich dieses befindet. Ein Schemata kann
von einer Normalform in eine andere (nächthöhere) überführt werden.
Dies lässt sich auch automatisiert durchführen. Jede dieser Normalformen
unterscheidet sich darin, wie die Abhängigkeiten zwischen den Tabellen
und Attributen sind. Je höher die Normalform, desto weniger Redundanz der Daten
(Wiederholung) ist vorhanden und desto weniger Inkosistenz und Anomalien gibt es.

\subsubsection{0. Normalform}

Die Daten liegen unstrukturiert, redundant und nicht-atomar vor.

\newcolumntype{a}{>{\columncolor{blue!20}}l}
\newcolumntype{d}{>{\columncolor{yellow!50}}l}

\begin{table}[H]
    \begin{tabular}{|a|l|l|l|}
    \hline
        CD\_ID & Albumtitel & Gründungsjahr & Titelliste \\ \hline
        11 & Sniff Dog - Los Angeles & 1995 & {1. Crazy 2. Love 3. Crack} \\ \hline
        12 & Jay-X - Goodbye New York & 1965 & {1. Imperial State of Math} \\ \hline
        13 & Sniff Dog - Dr. Drug & 1999 & {1. Weed Every Day} \\ \hline
    \end{tabular}
\end{table}

\clearpage

\subsubsection{1. Normalform}

Die Anforderungen an die 1. NF sind, dass alle Attribute atomar sind und
unnötige Spalten und Daten zusammengefasst werden, ohne das Informationen verloren gehen.

\begin{table}[H]
    \begin{tabular}{|a|l|l|l|a|l|}
    \hline
        CD\_ID & Albumtitel & Interpret & Gründungsjahr & Track & Titelliste \\ \hline
        11 & Los Angeles & Sniff Dog & 1993 & 1 & Crazy \\ \hline
        11 & Los Angeles & Sniff Dog & 1993 & 2 & Love \\ \hline
        11 & Los Angeles & Sniff Dog & 1993 & 3 & Crack \\ \hline
        12 & Goodbye New York & Jay-X & 1965 & 1 & Imperial State of Math \\ \hline
        13 & Dr. Drug & Sniff Dog & 1993 & 1 & Weed Every Day \\ \hline
    \end{tabular}
\end{table}

\subsubsection{2. Normalform}

Die Anforderungen an die 2. NF sind die an alle Vorigen und dass keine
nicht-schlüssel Attribute in einer Relation nur von einem Teil des Primärschlüssels
abhängig sind (jedes Attribut einer Relation ist voll funktional abhängig vom Primärschlüssel dieser).

\begin{table}[H]
    \begin{tabular}{|a|l|l|l|}
    \hline
        CD\_ID & Albumtitel & Interpret & Gründungsjahr \\ \hline
        11 & Los Angeles & Sniff Dog & 1993 \\ \hline
        12 & Goodbye New York & Jay-X & 1965 \\ \hline
        13 & Dr. Drug & Sniff Dog & 1993 \\ \hline
    \end{tabular}
\end{table}

\begin{table}[H]
    \begin{tabular}{|a|a|l|}
    \hline
        CD\_ID & Track & Titel \\ \hline
        11 & 1 & Crazy \\ \hline
        11 & 2 & Love \\ \hline
        11 & 3 & Crack \\ \hline
        12 & 1 & Imperial State of Math \\ \hline
        13 & 1 & Weed Every Day \\ \hline
    \end{tabular}
\end{table}

\clearpage

\subsubsection{3. Normalform}

Die Anforderungen an die 3. NF sind die an alle Vorigen und dass keine
nicht-schlüssel Attribute in einer Relation von anderen nicht-schlüssel Attributen
funktional abhängegen. Diese nicht gewollten Abhängigkeiten nennt man transistive Abhängigkeiten.

\begin{table}[H]
    \begin{tabular}{|a|l|d|}
    \hline
        CD\_ID & Albumtitel & Interpret\_ID \\ \hline
        11 & Los Angeles & 21 \\ \hline
        12 & Goodbye New York & 22 \\ \hline
        13 & Dr. Drug & 21 \\ \hline
    \end{tabular}
\end{table}

\begin{table}[H]
    \begin{tabular}{|a|l|l|}
    \hline
        Interpret\_ID & Interpret & Gründungsjahr \\ \hline
        21 & Sniff Dog & 1993 \\ \hline
        22 & Jay-X & 1965 \\ \hline
        21 & Sniff Dog & 1993 \\ \hline
    \end{tabular}
\end{table}

\begin{table}[H]
    \begin{tabular}{|a|a|l|}
    \hline
        CD\_ID & Track & Titel \\ \hline
        11 & 1 & Crazy \\ \hline
        11 & 2 & Love \\ \hline
        11 & 3 & Crack \\ \hline
        12 & 1 & Imperial State of Math \\ \hline
        13 & 1 & Weed Every Day \\ \hline
    \end{tabular}
\end{table}

\clearpage

\subsection{SQL}
\subsubsection{Struktur von SQL Abfragen}
