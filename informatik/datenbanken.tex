\section{Datenbanken}

\subsection{Begriffe bei Datenbanken}

In relationnellen Datenbanksystemen werden Daten als Entitäten modelliert
und strukturiert in sogenannten Tabellen (Relationen) gespeichert. Zwischen
diesen Tabellen werden dann auch zusammenhänge modelliert.

\vspace*{0.3cm}

Tabellen bestehen aus Zeilen (Datensätzen) und Spalten. Jede Spalte
steht dabei für ein Attribut der Tabelle, sodass jeder Datensatz
einen passenden Wert für jedes Attribut der Tabelle, also für jede Spalte, haben
muss. Jede Tabelle definiert ein oder mehrere Attribute, die den Primärschlüssel
darstellen. Der Primärschlüssels darf in keinen Datensätzen gleich sein, da dieser
zur eindeutigen Identifikation eines Datensatzes genutzt wird.
Außerdem gibt es noch Fremdschlüssel, die in Attributen einer Tabelle
gespeichert werden können. Dies sind Primärschlüssel einer anderen Tabelle,
wodurch Zusammenhänge zwischen Tabellen hergestellt werden können.

\vspace*{0.3cm}

Datensatz: Grau

Primärschlüssel: Blau

Attribut: Rot

Fremdschlüssel: Gelb

\begin{table}[H]
    \begin{tabular}{|l|l|l|}
    \hline
        CD\_ID & Albumtitel & Interpret\_ID \\ \hline
        \rowcolor{gray!20} 11 & Los Angeles & 21 \\ \hline
        12 & Goodbye New York & 22 \\ \hline
        13 & Dr. Drug & 21 \\ \hline
    \end{tabular}
\end{table}

\begin{table}[H]
    \begin{tabular}{|>{\columncolor{blue!20}}l|>{\columncolor{red!20}}l|>{\columncolor{yellow!50}}l|}
    \hline
        CD\_ID & Albumtitel & Interpret\_ID \\ \hline
        11 & Los Angeles & 21 \\ \hline
        12 & Goodbye New York & 22 \\ \hline
        13 & Dr. Drug & 21 \\ \hline
    \end{tabular}
\end{table}

\clearpage

\subsection{Entity-Relationship-Modelle (ER-Modelle)}

Ein Entity-Relationsip Modell soll Entitäten und Beziehungen eines Sachverhalts
darstellen und so z. B. die Erstellung einer Datenbank erleichtern.
Es gibt mehrere Arten zur Notation von Entity Relationship Modellen,
in diesem Fall wird die Chen Notation betrachtet.

\vspace*{1cm}

\adjustbox{scale=0.8}{
\begin{tikzpicture}[auto,node distance=1cm]
\node[entity] (node1) {Student}[grow=up,sibling distance=3.5cm]
    child[text width=2cm, align=center, level distance=2cm] {node[attribute] {Lastname}}
    child[text width=2cm, align=center, level distance=2cm] {node[attribute] {Firstname}}
    child[text width=2cm, align=center, level distance=2cm] {node[attribute] {\underline{ID}}};

\node[relationship] (rel1) [below = of node1] {Takes};
\node[attribute] (node4) [left = of rel1] {Hours};
\path (rel1) edge (node4);

\node[entity] (node2) [below = of rel1]{Class}[grow=down,sibling distance=3.5cm]
    child[text width=2cm, align=center, level distance=2cm] {node[attribute] {\underline{ID}}}
    child[text width=2cm, align=center, level distance=2cm] {node[attribute] {Subject}}
    child[text width=2cm, align=center, level distance=2cm] {node[attribute] {Teacher}};
\path (rel1) edge node {n} (node1) edge node {m}(node2);

\node[relationship] (rel2) [right = of node1] {Goes To};

\node[entity] (node3) [right = of rel2]{School}[grow=right, sibling distance=2.5cm]
    child[text width=2cm, align=center, level distance=2cm] {node[attribute] {\underline{ID}}}
    child[text width=2cm, align=center, level distance=2cm, xshift=2cm] {node[attribute] {Name}}
    child[text width=2cm, align=center, level distance=2cm] {node[attribute] {Type}};

\path (rel2) edge node {m} (node1) edge node {1}(node3);
\end{tikzpicture}
}

\vspace*{0.3cm}

\subsubsection{Erklärung der ER-Notation nach Chen}

\vspace*{0.3cm}

\begin{enumerate}
    \item Ein Rechteck stellt eine Entität dar.
    \item Ein Oval stellt ein Attribut einer Entität dar.
    \item Ein unterstrichenes Attribut ist Teil des Primärschlüssels.
    \item Eine Raute erläutert die Beziehung zweier Entitäten.
    \item Ein Oval an einer Raute kann der Beziehung Attribute zuordnen.
    \item An einer Beziehung zwischen zwei Entitäten sind die Kardinalitäten notiert.
\end{enumerate}

\clearpage

\subsubsection{Kardinalitäten}

Die Kardinälität einer Beziehung beschreibt, wie vielen Entitäten der anderen
Entität zugeordnet werden können. Beispielsweise kann ein Schüler nur auf 1 Schule 
gehen, aber auf eine Schule können viele Schüler gehen. Klassen besucht ein Schüler
hingegen mehrere und in eine Klasse gehen mehrere Schüler. Es gibt folgende Kardinalitäten,
wobei man bei jeder Kardinalität auch angeben kann, ob einer Zuordnung optional ist
oder wie viele Zuordnung es geben muss.

\begin{table}[H]
    \begin{tabular}{|l|l|l|}
    \hline
        Kardinälität & Entität A & Entität B \\ \hline
        Eins zu Eins & 1 & 1 \\ \hline
        Eins zu Vielen & 1 & m \\ \hline
        Viele zu Eins & m & 1 \\ \hline
        Viele zu Viele & m & n \\ \hline
        \end{tabular}
\end{table}

\subsection{Normalformen}

Bei der Erstellung eines Datenbankschemas in einer Relationellen Datenbank
(z. b. aus einem ER-Modell) unterscheidet man, je nach Eigentschaften des
Schematas, in welcher Normalform sich dieses befindet. Ein Schemata kann
von einer Normalform in eine andere (nächthöhere) überführt werden.
Dies lässt sich auch automatisiert durchführen. Jede dieser Normalformen
unterscheidet sich darin, wie die Abhängigkeiten zwischen den Tabellen
und Attributen sind. Je höher die Normalform, desto weniger Redundanz der Daten
(Wiederholung) ist vorhanden und desto weniger Inkosistenz und Anomalien gibt es.

\subsubsection{0. Normalform}

Die Daten liegen unstrukturiert, redundant und nicht-atomar vor.

\newcolumntype{a}{>{\columncolor{blue!20}}l}
\newcolumntype{d}{>{\columncolor{yellow!50}}l}

\begin{table}[H]
    \begin{tabular}{|a|l|l|l|}
    \hline
        CD\_ID & Albumtitel & Gründungsjahr & Titelliste \\ \hline
        11 & Sniff Dog - Los Angeles & 1995 & {1. Crazy 2. Love 3. Crack} \\ \hline
        12 & Jay-X - Goodbye New York & 1965 & {1. Imperial State of Math} \\ \hline
        13 & Sniff Dog - Dr. Drug & 1999 & {1. Weed Every Day} \\ \hline
    \end{tabular}
\end{table}

\clearpage

\subsubsection{1. Normalform}

Die Anforderungen an die 1. NF sind, dass alle Attribute atomar sind und
unnötige Spalten und Daten zusammengefasst werden, ohne das Informationen verloren gehen.

\begin{table}[H]
    \begin{tabular}{|a|l|l|l|a|l|}
    \hline
        CD\_ID & Albumtitel & Interpret & Gründungsjahr & Track & Titelliste \\ \hline
        11 & Los Angeles & Sniff Dog & 1993 & 1 & Crazy \\ \hline
        11 & Los Angeles & Sniff Dog & 1993 & 2 & Love \\ \hline
        11 & Los Angeles & Sniff Dog & 1993 & 3 & Crack \\ \hline
        12 & Goodbye New York & Jay-X & 1965 & 1 & Imperial State of Math \\ \hline
        13 & Dr. Drug & Sniff Dog & 1993 & 1 & Weed Every Day \\ \hline
    \end{tabular}
\end{table}

\subsubsection{2. Normalform}

Die Anforderungen an die 2. NF sind die an alle Vorigen und dass keine
nicht-schlüssel Attribute in einer Relation nur von einem Teil des Primärschlüssels
abhängig sind (jedes Attribut einer Relation ist voll funktional abhängig vom Primärschlüssel dieser).

\begin{table}[H]
    \begin{tabular}{|a|l|l|l|}
    \hline
        CD\_ID & Albumtitel & Interpret & Gründungsjahr \\ \hline
        11 & Los Angeles & Sniff Dog & 1993 \\ \hline
        12 & Goodbye New York & Jay-X & 1965 \\ \hline
        13 & Dr. Drug & Sniff Dog & 1993 \\ \hline
    \end{tabular}
\end{table}

\begin{table}[H]
    \begin{tabular}{|a|a|l|}
    \hline
        CD\_ID & Track & Titel \\ \hline
        11 & 1 & Crazy \\ \hline
        11 & 2 & Love \\ \hline
        11 & 3 & Crack \\ \hline
        12 & 1 & Imperial State of Math \\ \hline
        13 & 1 & Weed Every Day \\ \hline
    \end{tabular}
\end{table}

\clearpage

\subsubsection{3. Normalform}

Die Anforderungen an die 3. NF sind die an alle Vorigen und dass keine
nicht-schlüssel Attribute in einer Relation von anderen nicht-schlüssel Attributen
funktional abhängegen. Diese nicht gewollten Abhängigkeiten nennt man transistive Abhängigkeiten.

\begin{table}[H]
    \begin{tabular}{|a|l|d|}
    \hline
        CD\_ID & Albumtitel & Interpret\_ID \\ \hline
        11 & Los Angeles & 21 \\ \hline
        12 & Goodbye New York & 22 \\ \hline
        13 & Dr. Drug & 21 \\ \hline
    \end{tabular}
\end{table}

\begin{table}[H]
    \begin{tabular}{|a|l|l|}
    \hline
        Interpret\_ID & Interpret & Gründungsjahr \\ \hline
        21 & Sniff Dog & 1993 \\ \hline
        22 & Jay-X & 1965 \\ \hline
        21 & Sniff Dog & 1993 \\ \hline
    \end{tabular}
\end{table}

\begin{table}[H]
    \begin{tabular}{|a|a|l|}
    \hline
        CD\_ID & Track & Titel \\ \hline
        11 & 1 & Crazy \\ \hline
        11 & 2 & Love \\ \hline
        11 & 3 & Crack \\ \hline
        12 & 1 & Imperial State of Math \\ \hline
        13 & 1 & Weed Every Day \\ \hline
    \end{tabular}
\end{table}

\clearpage

\subsection{Datenbanken Anomalien und Inkonsistenz}

Es gibt 3 Arten von Anomlien in Datenbanken. Einfüge-, Änderungs- und Löschanomlie.
Diese treten meist auf, wenn das Datenbankschema unklar oder schlecht ist und Redundanzen
aufweist. Dies führt zur Inkonsistenz der Datenbank. Das heißt, es fehlen wichtige
Daten oder es gibt konfliktierende Datensätze, was die Integrität der Datenbank mindert.

\subsubsection{Einfügeanomlie}

Eine Einfügeanomalie tritt dann auf, wenn beim Einfügen eines Datensatzes
nicht alle Spalten mit einem Wert gefüllt werden können, z. B. weil der
Wert aufgrund des Sachzusammenhangs noch nicht bekannt ist.
Dadurch entstehen inkonsistenzen.
Beispielsweise hat ein Künstler sich noch nicht für einen Songnamen entschieden.

\begin{table}[H]
    \begin{tabular}{|a|a|l|}
    \hline
        CD\_ID & Track & Titel \\ \hline
        11 & 1 & Crazy \\ \hline
        11 & 2 & Love \\ \hline
        11 & 3 &  \\ \hline
        \end{tabular}
\end{table}

\subsubsection{Änderungsanomalie}

Ein Änderungsanomlie tritt dann auf, wenn Daten redundant vorliegen und bei
der Änderung z. B. eines Attributs nicht alle anderen Einträge auch geändert
werden. Dadurch gibt es konflikte der Daten, also Inkonsistenz.
Wird Beispielsweise der Albumname nicht für alle Songs geändert,
so weiß man nicht, welcher Name der Richtige ist.

\begin{table}[H]
    \begin{tabular}{|a|a|l|l|}
    \hline
        CD\_ID & Track & Albumtitel & Titel \\ \hline
        11 & 1 & Crazy & Los Angeles \\ \hline
        11 & 2 & Love  & Los Angeles\\ \hline
        11 & 3 & Crack & San Francisco \\ \hline
        \end{tabular}
\end{table}

\subsubsection{Löschanomlie}

Bei der Löschanomlie werden ungewollt Daten gelöscht. Dies
geschieht z. B. dann, wenn Beziehungen nicht in seperaten
Tabellen gespeichert sind oder wenn die Tabellen nicht korrekt
aufgebaut sind.

\clearpage

\subsection{SQL}

Das Aufsetzen eines Datenbank Servers und einer Datenbank sowie das Erstellen und
Modifizieren von Tabellen und die Grundlegenden Datentypen werden im folgenden
nicht behandelt. Es wird ein Fokus auf SQL-Abfragen gelegt.

\subsubsection{Auswahl von Tabellenspalten (SELECT, FROM)}

Wählt die angegebenen Spalten aus den Tabellen aus und zeigt diese an.

\begin{lstlisting}[language=sql]
SELECT [* | column, ...] FROM [table, ...];

SELECT vorname, alter FROM schueler;
\end{lstlisting}

\subsubsection{DISTINCT}

Wählt die angegebenen Spalten aus den Tabellen aus und zeigt diese an.
Kommen dabei in einer Spalte Werte mehrmals vor, so wird dieser Wert nur einmal angezeigt.

\begin{lstlisting}[language=sql]
SELECT DISTINCT [* | column, ...] FROM [table];

SELECT DISTINCT lehrer FROM kurse;
\end{lstlisting}

\subsubsection{WHERE}

Wählt nur Datensätze aus, die alle Bedingungen im \textbf{WHERE} Teil erfüllen.

\begin{lstlisting}[language=sql]
SELECT [* | column, ...]
FROM [table]
WHERE [condition [AND | OR] condition ...];

SELECT *
FROM kurse
WHERE lehrer LIKE 'Herr%' AND klasse > 10;
\end{lstlisting}

\subsubsection{Bedingungen in SQL}

Bedingungen im \textbf{WHERE} Teil lassen sich mit folgenden Operatoren formulieren.

\begin{lstlisting}[language=sql]
=, !=, >, <, >=, <=, NOT
LIKE <string> mit '_' (für 1 Zeichen) oder '%' (für 1 bis n Zeichen)
BETWEEN, IN, IS NULL
\end{lstlisting}

\begin{lstlisting}[language=sql]
[column] [operator] [column]
WHERE age < 5;
WHERE name LIKE 'a%';
WHERE NOT age = 3;
\end{lstlisting}

\clearpage

\subsubsection{ORDER BY}

Sortiert die Datensätze anhand ein oder mehr Spalten aufsteigend oder absteigend.

\begin{lstlisting}[language=sql]
SELECT [* | column, ...] FROM [table, ...] ORDER BY [column] [ASC | DESC];

SELECT vorname, alter FROM schueler ORDER BY alter asc;
\end{lstlisting}

\subsubsection{LIMIT}

Zeigt nur die ersten Datensätze an, bis das Limit an Zeilen erreicht ist.

\begin{lstlisting}[language=sql]
SELECT [* | column, ...] FROM [table, ...];

SELECT vorname, alter FROM schueler LIMIT 4;
\end{lstlisting}

\subsubsection{Aggregatfunktionen (AVG, COUNT, ...)}

\begin{lstlisting}[language=sql]
AVG() Arithmetisches Mittel der Werte
COUNT() Anzahl der Werte
MAX() Maximum der Werte
MIN() Minimum der Werte
SUM() Summer der Werte
\end{lstlisting}

\begin{lstlisting}[language=sql]
SELECT [AVG, ...]([* | column, ...]) FROM [table, ...];

SELECT COUNT(*) FROM schueler;
\end{lstlisting}

\subsubsection{AS (Alias)}

\begin{lstlisting}[language=sql]
SELECT [* | column [AS alias], ...] FROM [table, ...];

SELECT vorname AS name, alter FROM schueler LIMIT 4;
\end{lstlisting}

\subsubsection{Tabellen Zusammenführen}

Wenn man mehrere Tabellen im \textbf{FROM} Teil eingibt, so wird das Kreuzprodukt der
Tabellen gebildet. Das bedeutet, dass jede Zeile einer Tabelle einmal mit allen
Zeilen der anderen Tabellen so kombiniert wird, dass jede Wertekombination
aller Spalten einmal als Zeile vorkommt. Deswegen ist danach eine Filterung der
Daten nötig. Nutzt man nun \textbf{WHERE} auf diese neue Tabelle, so lassen
sich die Daten Tabellenübergreifend filtern (zusammenführen, ``joinen'').
Es können z. B. die Namen aller Schüler angezeigt werden, die im Kurs 'Dt 31L' sind

\begin{lstlisting}[language=sql]
SELECT vorname
FROM Schueler, Hatkurs, Kurse
WHERE 
    Schueler.id_nummer = Hatkurs.id_nummer AND
    Hatkurs.Kursnummer = Kurse.Kursnummer AND
    Kurse.Kursnummer = 'Dt 31L';
\end{lstlisting}

\subsubsection{GROUP BY}

\textbf{GROUP BY} fasst alle Datensätze zu einer Gruppe zusammen, wo die
Spalten den gleichen Wert haben. Dabei dürfen sich hinter \textbf{SELECT}
allerdings nur Attribute befinden, nach denen gruppiert wird, oder Aggregatfunktionen.

\begin{lstlisting}[language=sql]
SELECT name, plz FROM Schueler GROUP BY plz, name;
\end{lstlisting}

\subsubsection{HAVING}

\textbf{HAVING} bezieht sich auf eine Gruppe und nicht auf einen einzelnen Datensatz.
Dadurch lassen sich z. B. Gruppen aussortieren, die gewisse Bedingungen nicht erfüllen.

\begin{lstlisting}[language=sql]
SELECT TrainerID, Vorname, Nachname, SUM(Dauer)
FROM Trainer T, Kurs K
WHERE T.TrainerID = K.Trainer
GROUP BY TrainerID, Vorname, Nachname
HAVING SUM(Dauer) > 120 
\end{lstlisting}

\subsubsection{SQL Reihenfolge an Operationen}

\begin{lstlisting}[language=sql]
FROM -> JOIN -> WHERE -> GROUP BY -> HAVING -> SELECT -> COUNT, ... -> ORDER BY
\end{lstlisting}

\subsubsection{Strukturübersicht von SQL Abfragen}

\adjustbox{scale=1.3}{
\begin{lstlisting}[language=sql]
Abfragestruktur:
SELECT [DISTINCT | ALL] columns FROM tables
[WHERE bedingung1 (AND | OR) bedingung2 (AND | OR) ...]
[GROUP BY spalte1 , spalte2 , ...]
[HAVING bedingung1 (AND | OR) bedingung2 (AND | OR) ...]
[ORDER BY spalte1 [ASC | DESC], spalte2 [ASC | DESC], ...]
[LIMIT anzahl]

Operatoren für Berechnungen:
+, -, *, /

Operatoren für Vergleiche in Bedingungen: 
=, != , >, <, >=, <=, NOT
LIKE (mit den Platzhaltern _ und %) , BETWEEN, IN, IS NULL

Aggregatfunktionen: AVG( ), COUNT( ), MAX( ), MIN( ), SUM( ) 
\end{lstlisting}
}
