\section{Automaten}

\subsection{Deterministische Endliche Automaten (DEA)}

Deterministische Endliche Automaten werden vorallem dazu genutzt um zu prüfen,
ob Strings einem bestimmten Schema entsprechen, also z. B. bestimmte Buchstabenfolgen
enthalten sind (vgl. Reguläre Ausdrücke).
Diese bestehen aus \textbf{Zuständen} $Q$ und \textbf{Übergängen} mit einem
\textbf{Startzustand} $S$ und einem oder
mehreren \textbf{Endzuständen} $F$. Außerdem ist für jeden Automat
ein \textbf{Eingabealphabet} $\Sigma$ definiert, was alle Symbole enthält,
die in der Eingabe (dem String) vorkommen können.
Bei der graphischen Darstellung eines DEA wird allerdings meist nur das Eingabealphabet
$\Sigma$ zusätzlich mit angegeben, da sich der Rest aus der Darstellung ergibt.

\vspace*{0.5cm}

\begin{figure}[h]
    \centering
    \begin{tikzpicture}[shorten >=1pt,node distance=2cm,auto]
        \tikzstyle{every state}=[fill={rgb:black,1;white,10}]
      
        \node[state,initial]   (q_0)                      {$s_0$};
        \node[state]           (q_1) [right of=q_0]     {$s_1$};
        \node[state,accepting] (q_2) [right of=q_1]  {$s_2$};
        
        \path[->]
        (q_0)   edge              node {1} (q_1)
        (q_1) edge [loop above]  node {0} (q_1)
              edge [bend left]  node {1} (q_2)
        (q_2) edge [loop above]  node {0} (q_2)
              edge [bend left]  node {1} (q_1);
    \end{tikzpicture}
    \caption*{\textbf{DEA um gerade Anzahlen von Nullen zu erkennen}}
\end{figure}

\vspace*{-2cm}

\Large
\begin{align*}
    Q & = \{s_0, s_1, s_2\} \\
    \Sigma & = \{0, 1\} \\
    S & = s_0 \\
    F & = \{s_2\}
\end{align*}
\normalsize

Dieser DEA erkennt bei binären Daten (Eingabealphabet aus 0 und 1), ob die Anzahl
der Nullen gerade ist. Der Endzustand wird nur bei geraden Anzahlen erreicht.
Somit eignet sich der Automat, diese zu erkennen.
Ein Pfeil beschreibt einen Zustandsübergang. In einem bestimmten Zustand erfolgt
der Übergang an einem Pfeil, wenn das aktuelle Eingabesymbol zum Symbol über dem Pfeil passt.
Gibt es für diesen Zustand keinen Pfeil mit dem aktuellen Symbol, so schlägt der Vorgang
Fäll und ein Fehlerzustand wird erreicht. Dieser wird meist nicht dargestellt, was bedeutet
dass alle nicht definierten Übergänge in diesen führen (Wort muss mit einer 1 beginnen).

\clearpage

Für die Eingabe \textbf{10110} folgt eine Ablehnung.

\Large
\begin{equation*}
    s_0 \overset{1}{\rightarrow} s_1 \overset{0}{\rightarrow} s_1 \overset{1}{\rightarrow}
    s_2 \overset{1}{\rightarrow} s_1 \overset{0}{\rightarrow} s_1 
\end{equation*}
\normalsize

Die Eingabe \textbf{10111} wird hingegen korrekt akzeptiert.

\vspace*{-0.9cm}

\Large
\begin{equation*}
    s_0 \overset{1}{\rightarrow} s_1 \overset{0}{\rightarrow} s_1 \overset{1}{\rightarrow}
    s_2 \overset{1}{\rightarrow} s_1 \overset{1}{\rightarrow} s_2 
\end{equation*}
\normalsize

\clearpage

\subsection{Deterministische Keller Automaten (DKA)}

Deterministische Keller Automaten unterscheiden sich von DEAs dadurch,
dass diese einen Keller haben. Keller bedeutet in diesem Sinne Stapel (Stack),
als einen Speicher (Gedächtnis), wodurch im Gegensatz zum DEA einige
weitere Anwendungen möglich sind.
Auch der DKA wird mit \textbf{Zuständen} $Q$, \textbf{Übergängen},
\textbf{Startzustand} $S$,\textbf{Endzuständen} $F$, \textbf{Eingabealphabet} $\Sigma$
definiert. Zusätzlich wird bei den Übergängen aber noch das aktuelle Zeichen oben auf
dem Keller berücksichtigt und es können bei einem Übergang auch Dinge oben auf den Keller
gelegt werden. Auch gibt es noch ein \textbf{Kelleralphabet} $\Gamma$ und ein
\textbf{Kellerstartsymbol} $Z$.

\begin{figure}[h]
    \centering
    \begin{tikzpicture}[shorten >=1pt,node distance=3cm,auto]
        \tikzstyle{every state}=[fill={rgb:black,1;white,10}]
      
        \node[state,initial]   (q_0)                      {$s_0$};
        \node[state]           (q_1) [right of=q_0]     {$s_1$};
        \node[state]           (q_2) [right of=q_1]  {$s_2$};
        \node[state,accepting] (q_3) [right of=q_2]  {$s_3$};

        \path[->]
        (q_0) edge               node {(\#,0):n\#} (q_1)
        (q_1) edge [loop above]  node {(n,0):nn} (q_1)
              edge               node {(n,1):$\varepsilon$} (q_2)
        (q_2) edge [loop above]  node {(n,1):$\varepsilon$} (q_2)
        (q_2) edge               node {(\#,$\varepsilon$):$\varepsilon$} (q_3);
    \end{tikzpicture}
    \caption*{\textbf{DKA um gleiche Anzahlen von Nullen und Einsen zu erkennen}}
\end{figure}

\Large
\begin{align*}
    Q & = \{s_0, s_1, s_2\} \\
    \Sigma & = \{0, 1\} \\
    \Gamma & = \{\#, n\} \\
    S & = s_0 \\
    Z & = \# \\
    F & = \{s_2\}
\end{align*}
\normalsize

Dieser DKA erkennt, ob auf eine Anzahl Nullen die gleiche Anzahl an Einsen folgt.
Dies kann z. B. für die Eingaben \textbf{000111}, \textbf{00011} und \textbf{0001111}
durchgegangen werden.

\clearpage

Hier wird die Eingabe akzeptiert.

\begin{table}[h]
    \begin{tabular}{|l|l|l|}
    \hline
    Eingabe & Keller & Zustand \\ \hline
    000111 & \# & $s_0$ \\ \hline
    00111 & n\# & $s_1$ \\ \hline
    0111 & nn\# & $s_1$ \\ \hline
    111 & nnn\# & $s_1$ \\ \hline
    11 & nn\# & $s_2$ \\ \hline
    1 & n\# & $s_2$ \\ \hline
    - & \# & $s_2$ \\ \hline
    - & - & $s_3$ \\ \hline
    \end{tabular}
\end{table}

Die Eingabe ist am Ende leer, doch kein Übergang ohne Eingabe mit $n$
auf dem Keller existiert. Es gibt einen Fehler.

\begin{table}[h]
    \begin{tabular}{|l|l|l|}
    \hline
    Eingabe & Keller & Zustand \\ \hline
    00011 & \# & $s_0$ \\ \hline
    0011 & n\# & $s_1$ \\ \hline
    011 & nn\# & $s_1$ \\ \hline
    11 & nnn\# & $s_1$ \\ \hline
    1 & nn\# & $s_2$ \\ \hline
    - & n\# & $s_2$ \\ \hline
    \multicolumn{3}{|l|}{Fehler} \\ \hline
    \end{tabular}
\end{table}

Der Endzustand ist erreicht, doch die Eingabe ist noch nicht leer.
Doch es gibt keine Übergänge vom Endzustand. Es gibt einen Fehler.

\begin{table}[h]
    \begin{tabular}{|l|l|l|}
    \hline
    Eingabe & Keller & Zustand \\ \hline
    0001111 & \# & $s_0$ \\ \hline
    001111 & n\# & $s_1$ \\ \hline
    01111 & nn\# & $s_1$ \\ \hline
    1111 & nnn\# & $s_1$ \\ \hline
    111 & nn\# & $s_2$ \\ \hline
    11 & n\# & $s_2$ \\ \hline
    1 & \# & $s_2$ \\ \hline
    1 & - & $s_3$ \\ \hline
    \multicolumn{3}{|l|}{Fehler} \\ \hline
\end{tabular}
\end{table}

\clearpage

\subsection{Mealy Automaten}

Im Gegensatz zu DEA und DKA können Mealy Automaten dazu genutzt werden,
Codierungen und Umwandlungen durchzuführen, da diese eine Ausgabe besitzen.
Auch Mealy Automaten werden mit \textbf{Zuständen} $Q$, \textbf{Übergängen},
\textbf{Startzustand} $S$ und \textbf{Eingabealphabet} $\Sigma$
definiert. Allerdings gibt es \textbf{keinen Endzustand}.
Zusätzlich hat der Mealy Automat noch ein \textbf{Ausgabealphabet} $\Lambda$.
Bei jedem Übergang kann ein Symbol dieses Alphabets zur Ausgabe hinzugefügt werden.

\begin{figure}[h]
    \centering
    \begin{tikzpicture}[shorten >=1pt,node distance=3cm,auto]
        \tikzstyle{every state}=[fill={rgb:black,1;white,10}]
      
        \node[state,initial]   (q_0)                      {$s_0$};
        \node[state]           (q_1) [right of=q_0]     {$s_1$};
        
        \path[->]
        (q_0) edge [bend left]   node {0 / $\varepsilon$} (q_1)
        edge [loop above] node {1 / $\varepsilon$} (q_0)
        (q_1) edge [bend left]  node {1 / a} (q_0)
        edge [loop above] node {0 / $\varepsilon$} (q_1);
    \end{tikzpicture}
    \caption*{\textbf{Mealy Automat zur Transformation von binären Daten}}
\end{figure}

\vspace*{-2cm}

\Large
\begin{align*}
    Q & = \{s_0, s_1\} \\
    \Sigma & = \{0, 1\} \\
    \Lambda & = \{a\} \\
    S & = s_0 \\
\end{align*}
\normalsize

Dieser Mealy Automat hat in der Ausgabe ein $a$ für jede Folge von \textbf{01} in
der Eingabe. Für eine Eingabe von \textbf{010111} wäre die Ausgabe \textbf{aa}.
Existiert für einen Zustand kein passender Übergang, so gibt es auch beim Mealy Automaten
einen Fehler.

\begin{table}[h]
    \begin{tabular}{|l|l|l|}
    \hline
    Eingabe & Ausgabe & Zustand \\ \hline
    010111 & - & $s_0$ \\ \hline
    10111 & - & $s_1$ \\ \hline
    0111 & a & $s_0$ \\ \hline
    111 & a & $s_1$ \\ \hline
    11 & aa & $s_0$ \\ \hline
    1 & aa & $s_0$ \\ \hline
    - & aa & $s_0$ \\ \hline
\end{tabular}
\end{table}