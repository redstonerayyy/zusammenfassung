\section{Automaten}

\subsection{Deterministische Endliche Automaten (DEA)}

Deterministische Endliche Automaten werden vorallem dazu genutzt um zu prüfen,
ob Strings einem bestimmten Schema entsprechen, also z. B. bestimmte Buchstabenfolgen
enthalten sind (vgl. Reguläre Ausdrücke).
Diese bestehen aus \textbf{Zuständen} $Q$ und \textbf{Übergängen} mit einem
\textbf{Startzustand} $S$ und einem oder
mehreren \textbf{Endzuständen} $F$. Außerdem ist für jeden Automat
ein \textbf{Eingabealphabet} $\Sigma$ definiert, was alle Symbole enthält,
die in der Eingabe (dem String) vorkommen können.
Bei der graphischen Darstellung eines DEA wird allerdings meist nur das Eingabealphabet
$\Sigma$ zusätzlich mit angegeben, da sich der Rest aus der Darstellung ergibt.

\vspace*{0.5cm}

\begin{figure}[h]
    \centering
    \begin{tikzpicture}[shorten >=1pt,node distance=2cm,auto]
        \tikzstyle{every state}=[fill={rgb:black,1;white,10}]
      
        \node[state,initial]   (q_0)                      {$s_0$};
        \node[state]           (q_1) [right of=q_0]     {$s_1$};
        \node[state,accepting] (q_2) [right of=q_1]  {$s_2$};
        
        \path[->]
        (q_0)   edge              node {1} (q_1)
        (q_1) edge [loop above]  node {0} (q_1)
              edge [bend left]  node {1} (q_2)
        (q_2) edge [loop above]  node {0} (q_2)
              edge [bend left]  node {1} (q_1);
    \end{tikzpicture}
    \caption*{\textbf{DEA um gerade Anzahlen von Nullen zu erkennen}}
\end{figure}

\vspace*{-2cm}

\Large
\begin{align*}
    Q & = \{s_0, s_1, s_2\} \\
    \Sigma & = \{0, 1\} \\
    S & = s_0 \\
    F & = \{s_2\}
\end{align*}
\normalsize

Dieser DEA erkennt bei binären Daten (Eingabealphabet aus 0 und 1), ob die Anzahl
der Nullen gerade ist. Der Endzustand wird nur bei geraden Anzahlen erreicht.
Somit eignet sich der Automat, diese zu erkennen.
Ein Pfeil beschreibt einen Zustandsübergang. In einem bestimmten Zustand erfolgt
der Übergang an einem Pfeil, wenn das aktuelle Eingabesymbol zum Symbol über dem Pfeil passt.
Gibt es für diesen Zustand keinen Pfeil mit dem aktuellen Symbol, so schlägt der Vorgang
Fäll und ein Fehlerzustand wird erreicht. Dieser wird meist nicht dargestellt, was bedeutet
dass alle nicht definierten Übergänge in diesen führen (Wort muss mit einer 1 beginnen).
Für die Eingabe \textbf{10110} folgt eine Ablehnung.

\Large
\begin{equation*}
    s_0 \overset{1}{\rightarrow} s_1 \overset{0}{\rightarrow} s_1 \overset{1}{\rightarrow}
    s_2 \overset{1}{\rightarrow} s_1 \overset{0}{\rightarrow} s_1 
\end{equation*}
\normalsize

Die Eingabe \textbf{10111} wird hingegen korrekt akzeptiert.

\vspace*{-0.9cm}

\Large
\begin{equation*}
    s_0 \overset{1}{\rightarrow} s_1 \overset{0}{\rightarrow} s_1 \overset{1}{\rightarrow}
    s_2 \overset{1}{\rightarrow} s_1 \overset{1}{\rightarrow} s_2 
\end{equation*}
\normalsize

\subsection{Deterministische Keller Automaten (DKA)}

image
explanation

\subsection{Mealy Automat}

image
exlanation
