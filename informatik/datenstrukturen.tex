\section{Datenstrukturen}

Datenstrukturen dienen der strukturierten Speicherung von Daten, meist im Arbeitsspeicher.
Es soll schneller Zugriff und dass einfache Anwendungen von Algorithmen ermöglicht werden,
wobei es wichtig ist, dass die Datenstruktur zu den Daten passt.
Alle Beschreibungen der Datenstrukturen beziehen sich auf die Vorgaben des
Kerncurriculums Informatik.

\subsection{Statische Reihung}

Die statische Reihung (vgl. Array) speichert Daten eines Datentyps und hat
eine feste Länge. Der Zugriff auf die Daten erfolgt indexbasiert und
mit einer Zugriffszeit von $O(1)$.

\subsection{Dynamische Reihung}

\begin{table}[H]
    \begin{tabular}{|p{0.50\linewidth}|p{0.50\linewidth}|}
    \hline
    DynArray(): DynArray & Eine leere dynamische Reihung wird angelegt. \\ \hline
    isEmpty(): Wahrheitswert & Gibt True zurück, wenn die Reihung leer ist. \\ \hline
    getItem(Ganzzahl index): Inhalt & Gibt das Element am Index zurück. \\ \hline
    append(Inhalt inhalt) & Fügt ein Element am Ende hinzu. \\ \hline
    insertAt(Ganzzahl index, Inhalt inhalt) & Fügt ein Element am Index ein. Die Elemente rechts davon rücken um 1 Position nach rechts. \\ \hline
    setItem(Ganzzahl index, Inhalt inhalt) & Ersetzt das Element am Index. \\ \hline
    delete(Ganzzahl index) & Löscht das Element am Index, die anderen Elemente rücken nach links. \\ \hline
    getLength(): Ganzzahl & Gibt die Länge zurück \\ \hline
    \end{tabular}
\end{table}
























\subsection{Schlange}
\subsection{Stapel}
\subsection{Binärbaum}
\subsubsection{Traversierung von Binärbäumen}
\subsubsection*{Pre-Order}
\subsubsection*{In-Order}
\subsubsection*{Post-Order}
