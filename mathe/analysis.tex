\section{Ableitungen}

\subsection{Ableitungsregeln}

\subsubsection{Konstantenregel}

Die Ableitung einer Konstante ist 0 (und wird dann weggelassen).
Die Ableitung von $x$ ist 1.

\vspace*{-1.3cm}
\begin{align*}
    f(x) & = 3x + 5 \\
    f'(x) & = 3
\end{align*}

\subsubsection{Summenregel}
\subsubsection{Differenzregel}
\subsubsection{Potenzregel}
\subsubsection{Faktorregel}

\section*{Kettenregel}

Die Kettenregel ist hilfreich, wenn man eine Funktion ableiten muss, die eine Potenz enthält.
Sie zeigt, wie man eine Funktion abzuleiten kann, die als ihren Parameter eine weitere Funktion hat.

Als Beispiel:

$f(x) = (2x - 1)^3$

Die Funktion lässt sich nun in eine äußere und innere Funktion aufteilen.

$v(x) = 2x - 1$

$u(x) = x^3$

$f(x)$ lässt sich deshalb auch als $f(x) = u(v(x))$ schreiben.
Die Kettenregel besagt nun, das diese Funktion auf folgendem Weg aufgeleitet werden kann.

$f(x) = u(v(x))$

$f'(x) = v'(x) \cdot u'(v(x))$

Somit ist die Ableitung der Ursprungsfunktion

$f'(x) = 2 \cdot 3(2x - 1)^2 = 6(2x - 1)^2$

Diese setzt sich zusammen aus den Ableitungen von $v(x)$ und $u(x)$ der Ursprungsfunktion $v(x)$. 

$v'(x) = 2$

$u'(x) = 3x^2$

\section*{Produktregel}
Die Produktregel vereinfacht das Ableiten bei Funktionen, in denen 2 Terme
miteinander Multipliziert werden, die sich als eigene Funktionen schreiben lassen.
Als Beispiel:

$f(x) = (x^2 - 1.5) \cdot (0.5x - x^2)$

Die Funktion $f(x)$ lässt sich nun aufteilen in

$u(x) = x^2 - 1.5$

$v(x) = 0.5x - x^2$

Somit lässt sich $f(x)$ als $f(x) = u(x) \cdot v(x)$ schreiben.
Die Produktregel besagt, dass bei einer Funktion dieser Art gilt

$f(x) = u(x) \cdot v(x)$

$f'(x) = u'(x) \cdot v(x) + u(x) \cdot v'(x)$

Somit ist die Ableitung der Ursprungsfunktion

$f'(x) = 2x \cdot (0.5x - x^2) + (x^2 - 1.5) \cdot (0.5 - 2x)$

\section*{Quotientenregel}


\subsection{Nullstellen, Vielfachheit}

\section*{Nullstellen}
Die Nullstellen der Ursprungsfunktion können in mehreren Verfahren oder Kombination dieser bestimmt werden. Diese sind
\begin{itemize}
	\item Umformung(z. B. Ausklammern)
	\item pq-Formel
	\item Taschenrechner(solve)
	\item Polynomdivision(nicht behandelt)
\end{itemize}

Da die Umformung schwer einzugrenzen ist und die Lösung mit dem Taschenrechner einfach, wird nur die pq-Formel betrachtet.

Die pq-Formel lässt sich nur auf Quadratische Funktionen bzw. Funktionen des 2. Grades anwenden.
Dabei muss der Vorfaktor $a$ aus der Normalform $ax^2 + bx + c$ 1 sein.
$b$ wird im Folgenden als $p$ bezeichnet, $c$ als $q$.

Die pq-Formel lautet: \newline

$\mathlarger{x_{1,2} = -\frac{p}{2} \pm \sqrt[]{(\frac{p}{2})^2 - q}}$ \newline

Das Einsetzen ergibt in die Formel ergibt dann die Nullstellen der Funktion.

Wenn man von der Funktion $f(x) = x^2 + 4x + 2$ ausgeht, dann ist die zugehörige pq-Formel \newline

$\mathlarger{x_{1,2} = -\frac{4}{2} \pm \sqrt[]{(\frac{4}{2})^2 - 2}}$ \newline

Diese lässt sich vereinfachen zu \newline

$\mathlarger{x_{1,2} = -2 \pm \sqrt[]{4 - 2}}$

Ist allerdings die Summe unter der Wurzel negativ, so gibt es keine Lösungen.

\section*{y-Achsenabschnitt}

Der y-Achsenabschnitt ist der Wert der Funktion an der Stelle $x = 0$.
Er bezeichnet denn Schnittpunkt des Graphen mit der y-Achse und lässt sich
durch Einsetzen von Null in die Ursprungsfunktion berechnen. 

\subsection{Extrempunkte}

\section*{Bestimmen von Extremstellen}
Die Extremstellen eines Graphen sind die höchsten oder niedrigsten Punkte des Graphen.
Dabei unterscheided man zwischen lokalen und globalen Minima und Maxima. Global bedeutet,
dass es auf den ganzen Graphen der höchste oder tiefste Punkt ist, lokal, das es nur in einem Bereich
der höchste oder tiefste Punkt ist.
\newline

Die Extremstellen einer Funktion $f(x)$ sind an den Stellen, wo $f'(x) = 0$ gilt, also wo
die Ursprungsfunktion keine Steigung hat. Außerdem gibt die 2. Ableitung, wenn man sie mit Null
gleichgesetzt, also $f''(x) = 0$, aufschluss darüber, ob es ein Hoch- oder Tiefpunkt ist.
Ist der Wert der 2. Ableitung an der Stelle x größer als $0$, so liegt ein Tiefpunkt vor, ist der Wert kleiner
als $0$, dann ist es ein Hochpunkt.
\newline

Somit lässt sich festhalten, dass die zur Berechnung der Extremstellen die 1. Ableitung der Funktion
mit $0$ gleichgesetzt werden muss. An den Stellen wo nun $f'(x) = 0$ ist, muss nun x in die 2. Ableitung
eingesetzt werden, um zu prüfen, ob es ein Hoch- oder Tiefpunkt ist.
Zum Schluss können noch durch Einsetzen in die Ursprungsfunktion die tatsächlichen Punkte
berechnet werden.
\newline

1. Ableitung gleich Null setzen: $f'(x) = 0$

In 2. Ableitung einsetzen: $f''(x) > 0$ oder $f''(x) < 0$

Punkte berechen: $E( x | f(x) )$

\subsection{Wendepunkte, Sattelpunkte}

\section*{Bestimmen von Wendepunkten}
Die Wendepunkte bei einem Graphen sind die Punkte, wo der Graph sein Krümmungsverhalten ändert.
Voraussetzung dafür ist, dass die 2. Ableitung an diesem Punkt gleich $0$ ist und die 3. Ableitung ungleich $0$.
\newline

Für einen Wendepunkt gilt also:

$f''(x) = 0$
$f'''(x) \neq 0$
\newline

Zur Bestimmung der Wendestellen müssen nun zunächst die Ableitungen gebildet werden und
die 2. Ableitung wird mit 0 gleichgesetzt. Anschließend überprüft man dann mit der 3. Ableitung,
ob es ein Wendepunkt ist.

\subsection{Symetrie}

\section*{Symmetrie}
Achsensymetrie(y-Achse) $f(x) = f(-x)$

Punktsymetrie(Ursprung) $f(x) = -f(-x)$

\subsection{Limes/Unendlichkeit}

\section*{Verhalten im Unendlichen}
Für $x$ gegen $\infty$ oder $-\infty$ lässt sich ein Wert bestimmen.
Wichtig, ob eine Funktion begrenzt oder unbegrenzt wächst.
Gesucht ist \[\lim _{x \to \infty} f(x) = L\] oder \[\lim _{x \to -\infty} f(x) = L\]

\subsection{Definitions, Wertebereich}

\section*{Definitionsbereich}
Eine Menge an Zahlen, meist aus einem der Zahlenräume oder mit Bedingung, z. B. $x > 0$. Für jede dieser Zahlen
ist $f(x)$ definiert und liefert einen Wert.

\section*{Wertebereich}
Bereich, in dem die Funktionswerte liegen können, also Maximalwert und Minimalwert von y.

\subsection{Monotonie, Krümmung}

\section*{Monotonie}
$f'(x) > 0 \rightarrow monoton steigent$

$f'(x) < 0 \rightarrow monoton fallend$

Monotonie muss für alle Intervalle bestimmt werden
Dafür gelten die Intervalle zwischen $\infty$, $-\infty$ und den Nullstellen der 1. Ableitung.
Hat $f'(x)$ also bei $x = 0$ die einzige Nullstelle, so müssen die Intervalle $]-\infty;0]$ und $[0;\infty[$
betrachtet werden.

\section*{Krümmung}
Vorgehensweise wie bei Monotonie, nur mit 2. Ableitung

$f''(x) > 0 \rightarrow links gekruemmt$

$f''(x) < 0 \rightarrow rechts gekruemmt$

Krümmung muss für alle Intervalle bestimmt werden
Dafür gelten die Intervalle zwischen $\infty$, $-\infty$ und den Nullstellen der 2. Ableitung.
Hat $f''(x)$ also bei $x = 0$ die einzige Nullstelle, so müssen die Intervalle $]-\infty;0]$ und $[0;\infty[$
betrachtet werden.

\subsection{Tangentengleichung, Normalengleichung}

\subsection{Funktionsscharen}

Ein Funktionsschar ist eine Menge von Funktionen, die aus einer Funktion durch veränderung
eines Parameters geschieht, dem Scharparameter, auch k genannt.

Es lässt sich nun allgemein Untersuchen, wo die signifikaten Stellen des Graphen in Abhängigkeit
von k liegen.

Wenn man nun Beispielhaft die Funktion $f(x) = k \cdot x^3 - 2x^2$ nimmt mit dem Scharparameter k.

$f_{k}(x) = k \cdot x^3 - 2x^2$

$f'_{k}(x) = k \cdot 3x^2 - 4x^1$

$f''_{k}(x) = k \cdot 6x - 4$

Setzt man nun die 1. Ableitung mit 0 gleich, so erhält man die Nullstellen $x = 0$ und $x = \frac{4}{3k}$.
Die 2. Nullstelle lässt sich nun nach $k$ umformen, sodass man $k = \frac{4 - x}{3}$ erhält.
Bei den Nullstellen der 1. Ableitung muss also $k = \frac{4-x}{3}$ gelten.
Somit kann man in der Ursprungsfunktion $k$ mit $\frac{4-x}{3}$ ersetzen, wodurch man die Funktion
der Ortslinie erhält. Die Funktionsgleichung dieser ist \newline

$o(x) = \frac{4-x}{3} \cdot x^3 - 2x^2$ \newline

Setzt man die 2. Ableitung mit 0 gleich, so erhält man die Nullstelle $x = \frac{2}{3k}$
Dies entspricht also $k = \frac{2}{3x}$. Somit lässt dies in die Ursprungsfunktion einsetzen,
wodurch sich $w(x) = \frac{2}{3x} \cdot x^3 - 2x^2$


\section{GTR Todo}
